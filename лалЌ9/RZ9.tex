\documentclass[a4paper,12pt]{article} % This defines the style of your paper

\usepackage[top = 2.5cm, bottom = 2.5cm, left = 2.5cm, right = 2.5cm]{geometry} 

\usepackage[T2A]{fontenc}
\usepackage[utf8]{inputenc}
\usepackage[russian]{babel}

\usepackage{multirow} 
\usepackage{booktabs} 

\usepackage{graphicx} 

\usepackage{setspace}
\setlength{\parindent}{0in}

\usepackage{float}

\usepackage{amsmath}

\usepackage{fancyhdr}

\usepackage{pgfplots}
\pgfplotsset{compat=1.9}

\pagestyle{fancy} 

\fancyhf{} 

\lhead{\footnotesize Расчетное задание №9}

\rhead{\footnotesize Николаев Юрий} 

\cfoot{\footnotesize \thepage} 

\begin{document}

\thispagestyle{empty} 

\begin{tabular}{p{15.5cm}} 
НИУ МЭИ \\ А-13а-19  \\ Вариант 13 \\ Николаев Юрий\\
\hline 
\\
\end{tabular} 

\vspace*{0.3cm}

\begin{center} 
	{\Large \bf Расчетное задание №9} 
	\vspace{2mm}
\end{center}  

\vspace{0.4cm}


\section{Задание}
Вычислить нормы $\|\cdot\|_1$, $\|\cdot\|_E$, $\|\cdot\|_\infty$ матрицы $A$ и нормы $\|\cdot\|_1$, $\|\cdot\|_2$, $\|\cdot\|_\infty$ вектора $b$. Считая, что компоненты вектора $b$ получены в результате округления по дополнению, найти его относительную погрешность в каждой из трех указанных норм.

\begin{center}
\begin{tabular}{ | c c c | c | }
\hline
 \multicolumn{3}{| c |}{A} & b \\ \hline 
-2,693 & 2,013 & 2,284 & -3 \\
-2,487 & -2,574 & -0,792 & -3,87 \\
1,602 & 2,557 & 1,563 & 8 \\
\hline
\end{tabular}
\end{center}

\section{Решение}

\begin{enumerate}

\item Найдем нормы $\|\cdot\|_1$, $\|\cdot\|_E$, $\|\cdot\|_\infty$ матрицы $A$:

$\|A\|_1 = \max\limits_{{1 \leq j \leq m}} \sum\limits_{i = 1}^{m}|a_{i, j}| = \max\limits_{{1 \leq j \leq m}} (6,782; 7,144; 4,639) = 7,144$

$\|A\|_E = \sqrt{2.693^2 + 2.013^2 + 2.284^2 + 2.487^2 + 2.574^2 + 0.792^2 + 1.602^2 + 2.557^2 + 1.563^2} \approx 6,443$

$\|A\|_\infty = \max\limits_{{1 \leq i \leq m}} \sum\limits_{j = 1}^{m}|a_{i, j}| = \max\limits_{{1 \leq i \leq m}} (6,990; 5,853; 5,722)^T$


\item Найдем нормы $\|\cdot\|_1$, $\|\cdot\|_2$, $\|\cdot\|_\infty$ вектора $b$:

$\|b\|_1 = \sum\limits_{i = 1}^{m}|b_i| = 3 + 3,87 + 8 = 14,87$

$\|b\|_2 = \sqrt{\sum\limits_{i = 1}^{m}|b_i|^2} = \sqrt{9 + 14,9769 + 64} \approx 9,38$

$\|b\|_\infty = \max\limits_{{1 \leq i \leq m}}|b_i| = 8$

\newpage

\item Считая, что компоненты вектора $b$ получены в результате округления по дополнению, найдем его относительную погрешность в каждой из трех указанных норм:

$\Delta b = \|\overline{b} - b\|$

$\delta b = \frac{\Delta b}{\|\overline{b}\|}$

Абсолютные погрешности компонент вектора равны соответвенно: $5 \cdot 10^{-1}, 5 \cdot 10^{-3}, 5 \cdot 10^{-1}$. Тогда абсолютная погрешность вектора равна:

\begin{enumerate}
    \item в норме $\|b\|_1$: $\Delta b = 1,005 \approx 1$
    \item в норме $\|b\|_2$: $\Delta b \approx 0,7$
    \item в норме $\|b\|_\infty$: $\Delta b = 0,5$
\end{enumerate}

Тогда относительные погрешности соответственно для норм $\|b\|_1$, $\|b\|_2$, $\|b\|_\infty$ равны:

\begin{enumerate}
    \item в норме $\|b\|_1$: $\delta b \approx 6,7 \cdot 10^{-2}$
    \item в норме $\|b\|_2$: $\delta b \approx 7,5 \cdot 10^{-2}$
    \item в норме $\|b\|_\infty$: $\delta b \approx 6,3 \cdot 10^{-2} $
\end{enumerate}



\end{enumerate}

\end{document}