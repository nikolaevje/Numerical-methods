\documentclass[a4paper,12pt]{article} % This defines the style of your paper

\usepackage[top = 2.5cm, bottom = 2.5cm, left = 2.5cm, right = 2.5cm]{geometry} 

\usepackage[T2A]{fontenc}
\usepackage[utf8]{inputenc}
\usepackage[russian]{babel}

\usepackage{multirow} 
\usepackage{booktabs} 

\usepackage{graphicx} 

\usepackage{setspace}
\setlength{\parindent}{0in}

\usepackage{float}

\usepackage{amsmath}

\usepackage{fancyhdr}

\usepackage{pgfplots}
\pgfplotsset{compat=1.9}

\pagestyle{fancy} 

\fancyhf{} 

\lhead{\footnotesize Расчетное задание №8}

\rhead{\footnotesize Николаев Юрий} 

\cfoot{\footnotesize \thepage} 

\begin{document}

\thispagestyle{empty} 

\begin{tabular}{p{15.5cm}} 
НИУ МЭИ \\ А-13а-19  \\ Вариант 13 \\ Николаев Юрий\\
\hline 
\\
\end{tabular} 

\vspace*{0.3cm}

\begin{center} 
	{\Large \bf Расчетное задание №8} 
	\vspace{2mm}
\end{center}  

\vspace{0.4cm}


\section{Задание}
Решить систему уравнений $Ax = b$ методом прогонки.
УКАЗАНИЕ: промежуточные результаты вычислять с шестью знаками после запятой.

\begin{center}
\begin{tabular}{ | c c c c c | c | }
\hline
 \multicolumn{5}{| c |}{A} & b \\ \hline 
4 & -2 & 0 & 0 & 0 & 22 \\
-2 & 11 & 4 & 0 & 0 & 3 \\
0 & 1 & 12 & -5 & 0 & 31 \\
0 & 0 & 5 & 17 & 4 & 174 \\
0 & 0 & 0 & 4 & 7 & 46 \\
\hline
\end{tabular}
\end{center}

\section{Решение}

\begin{enumerate}

\item Вычислим прогоночные коэффициенты. Прямой ход прогонки:

$\gamma_1 = b_1 = 4 \Rightarrow 
\alpha_1 = -\frac{c_1}{\gamma_1} = \frac{1}{2}, 
\beta_1 = \frac{d_1}{\gamma_1} = \frac{11}{2}$

$\gamma_2 = b_2 + a_2\alpha_1 = 11 - 1 = 10  \Rightarrow 
\alpha_2 = -\frac{c_2}{\gamma_2} = -\frac{2}{5}, 
\beta_2 = \frac{d_2 - a_2\beta_1}{\gamma_2} = \frac{7}{5}$

$\gamma_3 = b_3 + a_3\alpha_2 = 12 - \frac{2}{5} = \frac{58}{5}  \Rightarrow 
\alpha_3 = -\frac{c_3}{\gamma_3} = \frac{25}{58}, 
\beta_3 = \frac{d_3 - a_3\beta_2}{\gamma_3} = \frac{74}{29}$

$\gamma_4 = b_4 + a_4\alpha_3 = 17 + \frac{125}{58} = \frac{1111}{58}  \Rightarrow 
\alpha_4 = -\frac{c_4}{\gamma_4} = -\frac{232}{1111}, 
\beta_4 = \frac{d_4 - a_4\beta_3}{\gamma_4} = \frac{9352}{1111}$

$\gamma_5 = b_5 + a_5\alpha_4 = 7 - \frac{928}{1111} = \frac{6849}{1111}  \Rightarrow 
\beta_5 = \frac{d_5 - a_5\beta_4}{\gamma_5} = \frac{13698}{6849} = 2$

\item Обратный ход метода прогонки:

\begin{center}
    $x_m = \beta_m$

    $x_i = \alpha_i x_{i+1} + \beta_i$

    $i = m - 1, m - 2, ..., 1$
\end{center}

$x_5 = \beta_5 = 2$

$x_4 = -\frac{2 \cdot 232}{1111} + \frac{9352}{1111} = \frac{8888}{1111} = 8$

$x_3 = \frac{8 \cdot 25}{58} + \frac{148}{58} = \frac{348}{58} = 6$

$x_2 = -\frac{6 \cdot 2}{5} + \frac{7}{5} = \frac{-5}{5} = -1$

$x_1 = -\frac{1}{2} + \frac{11}{2} = \frac{10}{2} = 5$

\textbf{Ответ: $x_1 = 5, x_2 = -1, x_3 = 6, x_4 = 8, x_5 = 2$}

\end{enumerate}

\end{document}