\documentclass[a4paper,12pt]{article} % This defines the style of your paper

\usepackage[top = 2.5cm, bottom = 2.5cm, left = 2.5cm, right = 2.5cm]{geometry} 

\usepackage[T2A]{fontenc}
\usepackage[utf8]{inputenc}
\usepackage[russian]{babel}

\usepackage{multirow} 
\usepackage{booktabs} 

\usepackage{graphicx} 

\usepackage{setspace}
\setlength{\parindent}{0in}

\usepackage{float}

\usepackage{amsmath}

\usepackage{fancyhdr}

\usepackage{pgfplots}
\pgfplotsset{compat=1.9}

\pagestyle{fancy} 

\fancyhf{} 

\lhead{\footnotesize Расчетное задание №10}

\rhead{\footnotesize Николаев Юрий} 

\cfoot{\footnotesize \thepage} 

\begin{document}

\thispagestyle{empty} 

\begin{tabular}{p{15.5cm}} 
НИУ МЭИ \\ А-13а-19  \\ Вариант 13 \\ Николаев Юрий\\
\hline 
\\
\end{tabular} 

\vspace*{0.3cm}

\begin{center} 
	{\Large \bf Расчетное задание №10} 
	\vspace{2mm}
\end{center}  

\vspace{0.4cm}


\section{Задание}
Вычислив норму обратной матрицы $A^{-1}$, оценить погрешность решения СЛАУ $Ax = b$ в каждой из трех указанных норм для найденных в задании 9 погрешностей вектора $b$.

\begin{center}
\begin{tabular}{ | c c c | c | }
\hline
 \multicolumn{3}{| c |}{A} & b \\ \hline 
-2,693 & 2,013 & 2,284 & -3 \\
-2,487 & -2,574 & -0,792 & -3,87 \\
1,602 & 2,557 & 1,563 & 8 \\
\hline
\end{tabular}
\end{center}

\section{Решение}

\begin{enumerate}

\item Обратная матрица $A^{-1}$:

\begin{equation*}
    A^{-1} = 
    \begin{pmatrix}
        -0,36 & 0,49 & 0,77 \\
        0,47 & -1,42 & -1,41 \\
        -0,4 & 1,82 & 2,15
    \end{pmatrix}
\end{equation*}

\item Найдем норму $\|\cdot\|_\infty$ матриц $A$, $A^{-1}$ и число обусловленности:

$\|A\|_\infty = \max\limits_{{1 \leq i \leq m}} \sum\limits_{j = 1}^{m}|a_{i, j}| = \max\limits_{{1 \leq i \leq m}} (6,990; 5,853; 5,722)^T = 6,99$

$\|A^{-1}\|_\infty = \max\limits_{{1 \leq i \leq m}} \sum\limits_{j = 1}^{m}|a_{i, j}| = \max\limits_{{1 \leq i \leq m}} (1,62; 3,30; 4,37)^T = 4,37$

$\nu_\delta = cond(A) = \|A\|\|A^{-1}\| = 30,5463$

\item Найдем нормы $\|\cdot\|_1$, $\|\cdot\|_2$, $\|\cdot\|_\infty$ вектора $b$:

$\|b\|_1 = \sum\limits_{i = 1}^{m}|b_i| = 3 + 3,87 + 8 = 14,87$

$\|b\|_2 = \sqrt{\sum\limits_{i = 1}^{m}|b_i|^2} = \sqrt{9 + 14,9769 + 64} \approx 9,38$

$\|b\|_\infty = \max\limits_{{1 \leq i \leq m}}|b_i| = 8$

\newpage

\item Относительные погрешности вектора $b$ соответственно для норм $\|b\|_1$, $\|b\|_2$, $\|b\|_\infty$ равны:

\begin{enumerate}
    \item в норме $\|b\|_1$: $\delta b \approx 6,7 \cdot 10^{-2}$
    \item в норме $\|b\|_2$: $\delta b \approx 7,5 \cdot 10^{-2}$
    \item в норме $\|b\|_\infty$: $\delta b \approx 6,3 \cdot 10^{-2} $
\end{enumerate}

\item Получим оценки погрешности решения СЛАУ в каждой из трех указанных норм по формуле $\delta (x^*) \leq \nu_\delta \delta (b^*)$:

\begin{enumerate}
    \item в норме $\|b\|_1$: $\delta (x^*) \leq 30,5463 \cdot 6,7 \cdot 10^{-2} \approx 2,05$
    \item в норме $\|b\|_2$: $\delta (x^*) \leq 30,5463 \cdot 7,5 \cdot 10^{-2} \approx 2,29$
    \item в норме $\|b\|_\infty$: $\delta (x^*) \leq 30,5463 \cdot 6,3 \cdot 10^{-2} \approx 1,92$
\end{enumerate}


\end{enumerate}

\end{document}