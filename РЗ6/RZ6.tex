\documentclass[a4paper,12pt]{article} % This defines the style of your paper

\usepackage[top = 2.5cm, bottom = 2.5cm, left = 2.5cm, right = 2.5cm]{geometry} 

\usepackage[T2A]{fontenc}
\usepackage[utf8]{inputenc}
\usepackage[russian]{babel}

\usepackage{multirow} 
\usepackage{booktabs} 

\usepackage{graphicx} 

\usepackage{setspace}
\setlength{\parindent}{0in}

\usepackage{float}

\usepackage{amsmath}

\usepackage{fancyhdr}

\usepackage{pgfplots}
\pgfplotsset{compat=1.9}

\pagestyle{fancy} 

\fancyhf{} 

\lhead{\footnotesize Расчетное задание №5}

\rhead{\footnotesize Николаев Юрий} 

\cfoot{\footnotesize \thepage} 

\begin{document}

\thispagestyle{empty} 

\begin{tabular}{p{15.5cm}} 
НИУ МЭИ \\ А-13а-19  \\ Вариант 13 \\ Николаев Юрий\\
\hline 
\\
\end{tabular} 

\vspace*{0.3cm}

\begin{center} 
	{\Large \bf Расчетное задание №5} 
	\vspace{2mm}
\end{center}  



\section{Задание}
Записать $LU$ разложение матрицы $A$ из задачи 5 (не проводя дополнительных расчетов). Используя полученное разложение, найти решение системы $Ax = d$.

\begin{center}
\begin{tabular}{| c |}
\hline
d \\ \hline 
104 \\
-462 \\
978 \\
-1968 \\
\hline
\end{tabular}
\end{center}

\section{Решение}

\begin{enumerate}

\item Запишем $LU$-разложение матрицы:

$
A = LU = \left(\begin{matrix}
1 & 0 & 0 & 0 \\
-5 & 1 & 0 & 0 \\
7 & 3 & 1 & 0 \\
-10 & -7 & -7 & 1 \\
\end{matrix}\right)
\left(\begin{matrix}
-2 & -2 & -8 & -10 \\ 
0 & 5 & 1 & -6 \\ 
0 & 0 & 7 & -9 \\
0 & 0 & 0 & -1 
\end{matrix}\right) =
\left(\begin{matrix}
-2 & -2 & -8 & -10\\ 
10 & 15 & 41 & 44\\ 
-14 & 1 & -46 & -97\\
20 & -15 & 24 & 204  
\end{matrix}\right)
$


\item Запишем систему как $LUx = d$, для удобства введем вспомогательны вектор $y = Ux$. Из $Ly = d$ найдем $y$:
\begin{equation*}
    Ly = 
    \begin{cases}
    y_1 = 104\\
    -5y_1 + y_2 = -462\\
    7y_1 + 3y_2 + y_3 = 978\\
    -10y_1 - 7y_2 - 7y_3 + y_4 = -1968
    \end{cases} 
    \Rightarrow
    \begin{cases}
    y_1 = 104\\
    y_2 = 520 - 462 = 58\\
    y_3 = 978 - 728 - 174 = 76\\
    y_4 = -1968 + 1040 + 406 + 532 = 10
    \end{cases}
\end{equation*}

\item Теперь вычислим $Ux = y$, таким образом, найдем решение:
\begin{equation*}
    Ux = 
    \begin{cases}
    -2x_1 - 2x_2 - 8x_3 - 10x_4 = 104\\
    5x_2 + x_3 - 6x_4 = 58\\
    7x_3 - 9x_4 = 76\\
    -x_4 = 10
    \end{cases} 
    \Rightarrow
    \begin{cases}
    x_1 = \frac{104 - 100 - 16}{-2} = 6\\
    x_2 = \frac{58 + 2 - 60}{5} = 0\\
    x_3 = \frac{76 - 90}{7} = -2\\
    x_4 = -10
    \end{cases}
\end{equation*}

\begin{equation*}
    Answer:
    \begin{cases}
    x_1 = 6\\
    x_2 = 0\\
    x_3 = -2\\
    x_4 = -10\\
    \end{cases}
\end{equation*}

\end{enumerate}
\end{document}