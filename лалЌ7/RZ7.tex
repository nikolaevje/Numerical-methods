\documentclass[a4paper,12pt]{article} % This defines the style of your paper

\usepackage[top = 2.5cm, bottom = 2.5cm, left = 2.5cm, right = 2.5cm]{geometry} 

\usepackage[T2A]{fontenc}
\usepackage[utf8]{inputenc}
\usepackage[russian]{babel}

\usepackage{multirow} 
\usepackage{booktabs} 

\usepackage{graphicx} 

\usepackage{setspace}
\setlength{\parindent}{0in}

\usepackage{float}

\usepackage{amsmath}

\usepackage{fancyhdr}

\usepackage{pgfplots}
\pgfplotsset{compat=1.9}

\pagestyle{fancy} 

\fancyhf{} 

\lhead{\footnotesize Расчетное задание №7}

\rhead{\footnotesize Николаев Юрий} 

\cfoot{\footnotesize \thepage} 

\begin{document}

\thispagestyle{empty} 

\begin{tabular}{p{15.5cm}} 
НИУ МЭИ \\ А-13а-19  \\ Вариант 13 \\ Николаев Юрий\\
\hline 
\\
\end{tabular} 

\vspace*{0.3cm}

\begin{center} 
	{\Large \bf Расчетное задание №7} 
	\vspace{2mm}
\end{center}  

\vspace{0.4cm}


\section{Задание}
Решить систему уравнений $Ax = b$ методом Холецкого.

\begin{center}
\begin{tabular}{ | c  c  c | c | }
\hline
 \multicolumn{3}{| c |}{A} & b \\ \hline 
36 & 0 & 48 & -504 \\
0 & 4 & 12 & -104 \\
48 & 12 & 116 & -1080 \\
\hline
\end{tabular}
\end{center}

\section{Решение}

\begin{enumerate}

\item Найдем разложение $LL^T$:

\begin{equation*}
    \begin{cases}
    36x_1 + 0x_2 + 48x_3 = -504\\
    0x_1 + 4x_2 + 12x_3 = -104\\
    48x_1 + 12x_2 + 116x_3 = -1080
    \end{cases}
\end{equation*}

$
LL^T = \left(\begin{matrix}
l_{11} & 0 & 0\\
l_{21} & l_{22} & 0\\
l_{31} & l_{32} & l_{33}\\
\end{matrix}\right)
\left(\begin{matrix}
l_{11} & l_{21} & l_{31}\\
0 & l_{22} & l_{32}\\
0 & 0 & l_{33}
\end{matrix}\right) =
\left(\begin{matrix}
36 & 0 & 48\\
0 & 4 & 12 \\
48 & 12 & 116
\end{matrix}\right)
$

1-ая строка:

$l_{11}^2 = 36 \Rightarrow l_{11} = 6$

$l_{21} \cdot l_{11} = 0 \Rightarrow l_{21} = \frac{0}{6} = 0$

$l_{31} \cdot l_{11} = 48 \Rightarrow l_{31} = \frac{48}{6} = 8$

2-ая строка:

$l_{21} \cdot l_{11} = 0 \Rightarrow l_{21} = \frac{0}{6} = 0$ - равенство было получено выше в силу симметрии.

$l_{21}^2 + l_{22}^2 = 4 \Rightarrow l_{22}^2 = 4 - 0 = 4 \Rightarrow l_{22} = 2$

$l_{21} \cdot l_{31} + l_{22} \cdot l_{32} = 12 \Rightarrow l_{32} = 6$

3-ья строка:

$l_{31}^2 + l_{32}^2 + l_{33}^2 = 116 \Rightarrow l_{33} = \sqrt{116 - 36 - 64} = 4$

\newpage

Получили $LL^T$ разложение:

\vspace{0.2cm}
$
LL^T = \left(\begin{matrix}
6 & 0 & 0\\
0 & 2 & 0\\
8 & 6 & 4\\
\end{matrix}\right)
\left(\begin{matrix}
6 & 0 & 8\\
0 & 2 & 6\\
0 & 0 & 4
\end{matrix}\right) =
\left(\begin{matrix}
36 & 0 & 48\\
0 & 4 & 12 \\
48 & 12 & 116
\end{matrix}\right)
$

\item Как в методе $LU$-разложения решаем две системы уравнений:

\begin{equation*}
    \begin{cases}
    6y_1 = -504\\
    2y_2 = -104\\
    8y_1 + 6y_2 + 4y_3 = -1080
    \end{cases}
    \Rightarrow
    y =
    \begin{cases}
    y_1 = -84\\
    y_2 = -52\\
    y_3 = -96
    \end{cases}
\end{equation*}

\begin{equation*}
    \begin{cases}
    6x_1 + 8x_3 = -84\\
    2x_2 + 6x_3 = -52\\
    4x_3 = -96
    \end{cases}
    \Rightarrow
    x =
    \begin{cases}
    x_1 = 18\\
    x_2 = 46\\
    x_3 = -24
    \end{cases}
\end{equation*}

\end{enumerate}
\begin{center}
\Large \textbf{Ответ:}
\begin{equation*}
    \begin{cases}
    x_1 = 18\\
    x_2 = 46\\
    x_3 = -24
    \end{cases}
\end{equation*}
\end{center}

\end{document}