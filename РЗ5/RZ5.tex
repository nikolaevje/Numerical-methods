\documentclass[a4paper,12pt]{article} % This defines the style of your paper

\usepackage[top = 2.5cm, bottom = 2.5cm, left = 2.5cm, right = 2.5cm]{geometry} 

\usepackage[T2A]{fontenc}
\usepackage[utf8]{inputenc}
\usepackage[russian]{babel}

\usepackage{multirow} 
\usepackage{booktabs} 

\usepackage{graphicx} 

\usepackage{setspace}
\setlength{\parindent}{0in}

\usepackage{float}

\usepackage{amsmath}

\usepackage{fancyhdr}

\usepackage{pgfplots}
\pgfplotsset{compat=1.9}

\pagestyle{fancy} 

\fancyhf{} 

\lhead{\footnotesize Расчетное задание №5}

\rhead{\footnotesize Николаев Юрий} 

\cfoot{\footnotesize \thepage} 

\begin{document}

\thispagestyle{empty} 

\begin{tabular}{p{15.5cm}} 
НИУ МЭИ \\ А-13а-19  \\ Вариант 13 \\ Николаев Юрий\\
\hline 
\\
\end{tabular} 

\vspace*{0.3cm}

\begin{center} 
	{\Large \bf Расчетное задание №5} 
	\vspace{2mm}
\end{center}  

\vspace{0.4cm}


\section{Задание}
Решить систему уравнений $Ax = b$ методом Гаусса (схема единственного деления).

\begin{center}
\begin{tabular}{ | c  c  c c | c | }
\hline
 \multicolumn{4}{| c |}{A} & b \\ \hline 
-2 & -2 & -8 & -10 & 60 \\
10 & 15 & 41 & 44 & -276 \\
-14 & 1 & -46 & -97 & 567 \\
20 & -15 & 24 & 204 & -1287 \\
\hline
\end{tabular}
\end{center}

\section{Решение}

\begin{enumerate}

\item Исключаем неизвестные из всех уравнений, кроме первого:

\begin{equation*}
    \begin{cases}
    -2x_1 - 2x_2 - 8x_3 - 10x_4 = 60\\
    10x_1 + 15x_2 + 41x_3 + 44x_4 = -276\\
    -14x_1 + x_2 - 46x_3 - 97x_4 = 567\\
    20x_1 - 15x_2 + 24x_3 + 204x_4 = -1287
    \end{cases}
\end{equation*}

$
A = \left(\begin{matrix} 
-2 & -2 & -8 & -10\\ 
10 & 15 & 41 & 44\\ 
-14 & 1 & -46 & -97\\
20 & -15 & 24 & 204 
\end{matrix}\left| 
\begin{matrix} 
60 \\ -276 \\ 567 \\ -1287 
\end{matrix}\right)\right.\ 
\sim~\ 
\begin{pmatrix} 
\mu_{2,1} = \frac{10}{-2} = -5\\ 
\mu_{3,1} = \frac{-14}{-2} = 7 \\ 
\mu_{4,1} = \frac{20}{-2} = -10
\end{pmatrix} \\
\sim~\ 
\left(\begin{matrix} 
-2 & -2 & -8 & -10\\ 
0 & 5 & 1 & -6\\ 
0 & 15 & 10 & -27\\
0 & -35 & -56 & 104 
\end{matrix}\left| 
\begin{matrix} 
60 \\ 24 \\ 147 \\ -687 
\end{matrix}\right)\right.\ 
$


\item Ведущий элемент 2-го шага - "5":

$
A = \left(\begin{matrix} 
-2 & -2 & -8 & -10\\ 
0 & 5 & 1 & -6\\ 
0 & 15 & 10 & -27\\
0 & -35 & -56 & 104 
\end{matrix}\left| 
\begin{matrix} 
60 \\ 24 \\ 147 \\ -687 
\end{matrix}\right)\right.\ 
\sim~\ 
\begin{pmatrix} 
\mu_{3,2} = \frac{15}{5} = 3\\ 
\mu_{4,2} = \frac{-35}{5} = -7
\end{pmatrix} \\
\sim~\ 
\left(\begin{matrix} 
-2 & -2 & -8 & -10\\ 
0 & 5 & 1 & -6\\ 
0 & 0 & 7 & -9\\
0 & 0 & -49 & 62 
\end{matrix}\left| 
\begin{matrix} 
60 \\ 24 \\ 75 \\ -519 
\end{matrix}\right)\right.\ 
$

\newpage

\item Ведущий элемент 3-го шага - "7":

$
A = \left(\begin{matrix} 
-2 & -2 & -8 & -10\\ 
0 & 5 & 1 & -6\\ 
0 & 15 & 10 & -27\\
0 & -35 & -56 & 104 
\end{matrix}\left| 
\begin{matrix} 
60 \\ 24 \\ 75 \\ -519 
\end{matrix}\right)\right.\ 
\sim~\ 
\begin{pmatrix} 
\mu_{4,3} = \frac{-49}{7} = -7
\end{pmatrix} \\
\sim~\ 
\left(\begin{matrix} 
-2 & -2 & -8 & -10\\ 
0 & 5 & 1 & -6\\ 
0 & 0 & 7 & -9\\
0 & 0 & 0 & -1 
\end{matrix}\left| 
\begin{matrix} 
60 \\ 24 \\ 75 \\ 6 
\end{matrix}\right)\right.\ 
$
\vspace{0.5cm}

\begin{equation*}
    \begin{cases}
    -2x_1 - 2x_2 - 8x_3 - 10x_4 = 60\\
    5x_2 + x_3 - 6x_4 = 24\\
    7x_3 - 9x_4 = 75\\
    -x_4 = 6
    \end{cases}
\end{equation*}

Привели матрицу к треугольному виду, перейдем ко второму этапу.

\item Обратный ход метода Гаусса (снизу вверх):

$x_4 = -6$

$x_3 = \frac{75 - 54}{7} = 3$

$x_2 = \frac{24 - 36 - 3}{5} = -3$

$x_1 = \frac{60 - 60 + 24 - 6}{-2} = -9$

\end{enumerate}
\begin{center}
\Large \textbf{Ответ:}
\begin{equation*}
    \begin{cases}
    x_1 = -9\\
    x_2 = -3\\
    x_3 = 3\\
    x_4 = -6
    \end{cases}
\end{equation*}
\end{center}

\end{document}