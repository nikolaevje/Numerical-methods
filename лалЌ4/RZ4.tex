\documentclass[a4paper,12pt]{article} % This defines the style of your paper

\usepackage[top = 2.5cm, bottom = 2.5cm, left = 2.5cm, right = 2.5cm]{geometry} 

\usepackage[T2A]{fontenc}
\usepackage[utf8]{inputenc}
\usepackage[russian]{babel}

\usepackage{multirow} 
\usepackage{booktabs} 

\usepackage{graphicx} 

\usepackage{setspace}
\setlength{\parindent}{0in}

\usepackage{float}

\usepackage{fancyhdr}

\usepackage{pgfplots}
\pgfplotsset{compat=1.9}

\pagestyle{fancy} 

\fancyhf{} 

\lhead{\footnotesize Расчетное задание №4}

\rhead{\footnotesize Николаев Юрий} 

\cfoot{\footnotesize \thepage} 

\begin{document}

\thispagestyle{empty} 

\begin{tabular}{p{15.5cm}} 
НИУ МЭИ \\ А-13а-19  \\ Вариант 13 \\ Николаев Юрий\\
\hline 
\\
\end{tabular} 

\vspace*{0.3cm}

\begin{center} 
	{\Large \bf Расчетное задание №4} 
	\vspace{2mm}
\end{center}  

\vspace{0.4cm}


\section{Задание}
Найти корень нелинейного уравнения $f(x) = 0$, локализованный на отрезке $[a, b]$, методом Ньютона с точностью $\varepsilon = 10^{-8}$.

$$f(x) = e^x - \sin x - 2$$
$$[a, b] = [0, 3]$$

\section{Решение}

\begin{enumerate}

\item Расчетная формула принимает вид:
$$x^{(k + 1)} = x^{(k)} - \frac{e^{x^{(k)}} - \sin x^{(k)} - 2}{e^{x^{(k)}} - \cos x^{(k)}}$$

\item В качестве начального приближения возьмем середину отрезка $[a, b]$: $x^{(0)} = 1,5$

Вычисляем первое приближение:
$$x^{(1)} = x^{(0)} - \frac{e^{x^{(0)}} - \sin x^{(0)} - 2}{e^{x^{(0)}} - \cos x^{(0)}} = 1,5 - \frac{4,48168907 - 0,99749499 - 2}{4,48168907 - 0,07073720} = 1,16352068$$

Проверяем критерий окончания итераций: $|x^{(1)} - x^{(0)}| \approx 0,33647932 > \varepsilon$.

Точность не достигнута, переходим ко второй итерации:
$$x^{(2)} = x^{(1)} - \frac{e^{x^{(1)}} - \sin x^{(1)} - 2}{e^{x^{(1)}} - \cos x^{(1)}} = 1,06263907$$

Проверяем критерий окончания итераций: $|x^{(2)} - x^{(1)}| \approx 0,10088161 > \varepsilon$.

Точность не достигнута, переходим к третьей итерации:
$$x^{(3)} = x^{(2)} - \frac{e^{x^{(2)}} - \sin x^{(2)} - 2}{e^{x^{(2)}} - \cos x^{(2)}} = 1,05418367$$

Проверяем критерий окончания итераций: $|x^{(3)} - x^{(2)}| \approx 0,0084554 > \varepsilon$.

\newpage

Точность не достигнута, переходим к четвертой итерации:
$$x^{(4)} = x^{(3)} - \frac{e^{x^{(3)}} - \sin x^{(3)} - 2}{e^{x^{(3)}} - \cos x^{(3)}} = 1,05412713$$

Проверяем критерий окончания итераций: $|x^{(4)} - x^{(3)}| \approx 5,7 * 10^{-6} > \varepsilon$.

Точность не достигнута, переходим к пятой итерации:
$$x^{(5)} = x^{(4)} - \frac{e^{x^{(4)}} - \sin x^{(4)} - 2}{e^{x^{(4)}} - \cos x^{(4)}} = 1,05412712$$

Проверяем критерий окончания итераций: $|x^{(5)} - x^{(4)}| \approx 10^{-8} \approx \varepsilon$.

Неравенство выполнено, следовательно, точность достигнута.

Представим результаты в виде таблицы:

\begin{center}
\begin{tabular}{ | c | c | c  | }
\hline
$k$ & $x^{(k)}$ & $|x^{(k)} - x^{(k - 1)}|$\\ \hline 
0 & 1,50000000 & \\
1 & 1,16352068 & $\approx 0,33647932$ \\
2 & 1,06263907 & $\approx 0,10088161$ \\
3 & 1,05418367 & $\approx 0,0084554$ \\
4 & 1,05412713 & $\approx 5,7 * 10^{-6}$ \\
5 & 1,05412712 & $\approx 10^{-8}$ \\
\hline
\end{tabular}
\end{center}
\vspace{0.5cm}

\end{enumerate}
\begin{center}
\Large \textbf{Таким образом, найденное значение корня:} $$\dot x = x^{(5)} \pm \varepsilon = 1,05412712 \pm 0,00000001.$$
\end{center}

\end{document}