\documentclass[a4paper,12pt]{article} % This defines the style of your paper

\usepackage[top = 2.5cm, bottom = 2.5cm, left = 2.5cm, right = 2.5cm]{geometry} 

\usepackage[T2A]{fontenc}
\usepackage[utf8]{inputenc}
\usepackage[russian]{babel}

\usepackage{multirow} 
\usepackage{booktabs} 

\usepackage{graphicx} 

\usepackage{setspace}
\setlength{\parindent}{0in}

\usepackage{float}

\usepackage{amsmath}

\usepackage{fancyhdr}

\usepackage{pgfplots}
\pgfplotsset{compat=1.9}

\pagestyle{fancy} 

\fancyhf{} 

\lhead{\footnotesize Расчетное задание №11}

\rhead{\footnotesize Николаев Юрий} 

\cfoot{\footnotesize \thepage} 

\begin{document}

\thispagestyle{empty} 

\begin{tabular}{p{15.5cm}} 
НИУ МЭИ \\ А-13а-19  \\ Вариант 13 \\ Николаев Юрий\\
\hline 
\\
\end{tabular} 

\vspace*{0.3cm}

\begin{center} 
	{\Large \bf Расчетное задание №11} 
	\vspace{2mm}
\end{center}  

\vspace{0.4cm}


\section{Задание}
Дана система уравнений $Ax = b$. Привести ее к виду, удобному для итераций, проверить выполнение достаточного условия сходимости указанных ниже методов. Выполнить три итерации по методу Якоби и три итерации по методу Зейделя. Определить, во сколько раз уменьшится норма невязки в каждом случае. Используя апостериорную оценку, вычислить погрешность приближенного решения, полученного на третьей итерации каждого метода.

УКАЗАНИЕ. Для обеспечения выполнения достаточного условия сходимости воспользоваться перестановкой строк в исходной системе уравнений.

\begin{center}
\begin{tabular}{ | c c c c | c | }
\hline
 \multicolumn{4}{| c |}{A} & b \\ \hline 
-2 & -8 & -3 & 66 & -629 \\
67 & 4 & 0 & -5 & 507 \\
0 & -10 & 86 & -1 & -562 \\
0 & 115 & 9 & -6 & -348 \\
\hline
\end{tabular}
\end{center}

\section{Решение}

\begin{enumerate}

\item Приведем матрицу к виду, удобному для итераций.

\begin{equation*}
    A =
    \begin{pmatrix}
        67 & 4 & 0 & -5 & 507 \\
        0 & 115 & 9 & -6 & -348 \\
        0 & -10 & 86 & -1 & -562 \\
        -2 & -8 & -3 & 66 & -629
    \end{pmatrix}
\end{equation*}

\begin{equation*}
\begin{cases}
    x_1 = -\frac{4x_2}{67} + \frac{5x_4}{67} + \frac{507}{67} \\
    x_2 = -\frac{9x_3}{115} + \frac{6x_4}{115} - \frac{348}{115} \\
    x_3 = \frac{10x_2}{86} + \frac{x_4}{86} - \frac{562}{86} \\
    x_4 = \frac{2x_1}{66} + \frac{8x_2}{66} + \frac{3x_3}{66} - \frac{629}{66} \\
\end{cases}
\end{equation*}

\begin{equation*}
    B =
    \begin{pmatrix}
        0 & -\frac{4}{67} & 0 & \frac{5}{67}  \\
        0 & 0 & -\frac{9}{115} & \frac{6}{115}  \\
        0 & \frac{10}{86} & 0 & \frac{1}{86}  \\
        \frac{2}{66} & \frac{8}{66} & \frac{3}{66} & 0 
    \end{pmatrix}
\end{equation*}

\newpage

\item Проверим выполнение достаточного условия сходимости метода Якоби и Зейделя.

Достаточное условие сходимости метода Якоби: $\|B\|_\infty \approx 0,1969 < 1$ - выполняется.

Достаточное условие сходимости метода Зейделя: $\|B_1\|_\infty + \|B_2\|_\infty \approx 0,1969 + 0,1343 \approx 0,3312 < 1$ - выполняется.

\item Выполним три итерации по методу Якоби.

\begin{equation*}
\begin{cases}
    x_1^{(n+1)} = -\frac{4x_2^{(n)}}{67} + \frac{5x_4^{(n)}}{67} + \frac{507}{67} \\
    x_2^{(n+1)} = -\frac{9x_3^{(n)}}{115} + \frac{6x_4^{(n)}}{115} - \frac{348}{115} \\
    x_3^{(n+1)} = \frac{10x_2^{(n)}}{86} + \frac{x_4^{(n)}}{86} - \frac{562}{86} \\
    x_4^{(n+1)} = \frac{2x_1^{(n)}}{66} + \frac{8x_2^{(n)}}{66} + \frac{3x_3^{(n)}}{66} - \frac{629}{66} \\
\end{cases}
\end{equation*}

$x^{(0)} = (\frac{507}{67}, -\frac{348}{115}, -\frac{562}{86}, -\frac{629}{66})^T \approx (7,567; -3,026; -6,535; -9,530)^T$

1 итерация:

$x_1^{(1)} = \frac{348 \cdot 4}{115 \cdot 67} - \frac{629 \cdot 5}{66 \cdot 67} + \frac{507}{67} \approx 7,037$

$x_2^{(1)} = \frac{562 \cdot 9}{86 \cdot 115} - \frac{629 \cdot 6}{66 \cdot 115} - \frac{348}{115} \approx -3,012$

$x_3^{(1)} = -\frac{348 \cdot 10}{115 \cdot 86} - \frac{629}{66 \cdot 86} - \frac{562}{86} \approx -6,998$

$x_4^{(1)} = \frac{507 \cdot 2}{67 \cdot 66} - \frac{348 \cdot 8}{115 \cdot 66} - \frac{562 \cdot 3}{86 \cdot 66} - \frac{629}{66} \approx -9,965$

$x^{(1)} = (7,037; -3,012; -6,998; -9,965)^T$

2 итерация:

$x_1^{(2)} = \frac{1}{67} \cdot (-4 \cdot (-3,012) - 5 \cdot 9,965 + 507) \approx 7,003$

$x_2^{(2)} = \frac{1}{115} \cdot (-9 \cdot (-6,998) - 6 \cdot 9,965 - 348) \approx -2,998$

$x_3^{(2)} = \frac{1}{86} \cdot (10 \cdot (-3,012) - \cdot 9,965 - 562) \approx -7,001$

$x_4^{(2)} = \frac{1}{66} \cdot (2 \cdot 7,037 - 8 \cdot 3,012 - 3 \cdot 6,998 - 629) \approx -10$

$x^{(2)} = (7,003; -2,998; -7,001; -10)^T$

3 итерация:

$x_1^{(3)} = \frac{1}{67} \cdot (-4 \cdot (-2,998) - 5 \cdot 10 + 507) \approx 6,99999$

$x_2^{(3)} = \frac{1}{115} \cdot (-9 \cdot (-7,001) - 6 \cdot 10 - 348) \approx -2,99992$

$x_3^{(3)} = \frac{1}{86} \cdot (10 \cdot (-2,998) - \cdot 10 - 562) \approx -7$

$x_4^{(3)} = \frac{1}{66} \cdot (2 \cdot 7,003 - 8 \cdot 2,998 - 3 \cdot 7,001 - 629) \approx -10$

$x^{(3)} = (6,99999; -2,99992; -7; -10)^T$

\newpage

\begin{equation*}
    r^{(0)} = Ax^{(0)} - b =
    \begin{pmatrix}
        542,535 \\
        -349,625 \\
        -522,220 \\
        -600,301 \\
    \end{pmatrix}
    -
    \begin{pmatrix}
        507 \\
        -348 \\
        -562 \\
        -629 \\
    \end{pmatrix}
    =
    \begin{pmatrix}
        35,535 \\
        -1,625 \\
        39,780 \\
        28,699 \\
    \end{pmatrix}
    , \|r^{(0)}\| = 39,780
\end{equation*}

\begin{equation*}
    r^{(1)} = Ax^{(1)} - b =
    \begin{pmatrix}
        509,256 \\
        -349,572 \\
        -561,743 \\
        -626,674 \\
    \end{pmatrix}
    -
    \begin{pmatrix}
        507 \\
        -348 \\
        -562 \\
        -629 \\
    \end{pmatrix}
    =
    \begin{pmatrix}
        2,256 \\
        -1,572 \\
        0,257 \\
        2,326 \\
    \end{pmatrix}
    , \|r^{(1)}\| = 2,326
\end{equation*}

\begin{equation*}
    r^{(2)} = Ax^{(2)} - b =
    \begin{pmatrix}
        507,209 \\
        -347,779 \\
        -562,106 \\
        -629,019 \\
    \end{pmatrix}
    -
    \begin{pmatrix}
        507 \\
        -348 \\
        -562 \\
        -629 \\
    \end{pmatrix}
    =
    \begin{pmatrix}
        0,209 \\
        0,221 \\
        -0,106 \\
        -0,019 \\
    \end{pmatrix}
    , \|r^{(2)}\| = 0,221
\end{equation*}

\begin{equation*}
    r^{(3)} = Ax^{(3)} - b =
    \begin{pmatrix}
        506,99965 \\
        -347,9908 \\
        -562,0008 \\
        -629,00062 \\
    \end{pmatrix}
    -
    \begin{pmatrix}
        507 \\
        -348 \\
        -562 \\
        -629 \\
    \end{pmatrix}
    =
    \begin{pmatrix}
        -0,00035 \\
        0,0092 \\
        -0,0008 \\
        -0,00062 \\
    \end{pmatrix}
    , \|r^{(3)}\| = 0,0092
\end{equation*}

$$\frac{\|r^{(0)}\|}{\|r^{(2)}\|} = \frac{39,780}{0,0092} \approx 4324$$

\item Выполним три итерации по методу Зейделя.

\begin{equation*}
\begin{cases}
    x_1^{(n+1)} = -\frac{4x_2^{(n)}}{67} + \frac{5x_4^{(n)}}{67} + \frac{507}{67} \\
    x_2^{(n+1)} = -\frac{9x_3^{(n)}}{115} + \frac{6x_4^{(n)}}{115} - \frac{348}{115} \\
    x_3^{(n+1)} = \frac{10x_2^{(n+1)}}{86} + \frac{x_4^{(n)}}{86} - \frac{562}{86} \\
    x_4^{(n+1)} = \frac{2x_1^{(n+1)}}{66} + \frac{8x_2^{(n+1)}}{66} + \frac{3x_3^{(n+1)}}{66} - \frac{629}{66} \\
\end{cases}
\end{equation*}

$x^{(0)} = (\frac{507}{67}, -\frac{348}{115}, \frac{10 \cdot (-3,026)}{86} -\frac{562}{86}, \frac{2 \cdot 7,567}{66} + \frac{8 \cdot (-3,026)}{66} + \frac{3 \cdot (-6,887)}{66} - \frac{629}{66})^T \approx $

$\approx (7,567; -3,026; -6,887; -9,981)^T$

1 итерация:

$x_1^{(1)} = \frac{1}{67} \cdot (-4 \cdot (-3,026) - 5 \cdot 9,981 + 507) \approx 7,003$

$x_2^{(1)} = \frac{1}{115} \cdot (-9 \cdot (-6,887) - 6 \cdot 9,981 - 348) \approx -3,008$

$x_3^{(1)} = \frac{1}{86} \cdot (10 \cdot (-3,008) - \cdot 9,981 - 562) \approx -7,001$

$x_4^{(1)} = \frac{1}{66} \cdot (2 \cdot 7,003 - 8 \cdot 3,008 - 3 \cdot 7,001 - 629) \approx -10,001$

$x^{(1)} = (7,003; -3,008; -7,001; -10,001)^T$

\newpage

2 итерация:

$x_1^{(2)} = \frac{1}{67} \cdot (-4 \cdot (-3,008) - 5 \cdot 10,001 + 507) \approx 7,0004$

$x_2^{(2)} = \frac{1}{115} \cdot (-9 \cdot (-7,001) - 6 \cdot 10,001 - 348) \approx -2,99997$

$x_3^{(2)} = \frac{1}{86} \cdot (10 \cdot (-3) - \cdot 10,001 - 562) \approx -7,00001$

$x_4^{(2)} = \frac{1}{66} \cdot (2 \cdot 7 - 8 \cdot 3 - 3 \cdot 7,00001 - 629) \approx -10$

$x^{(2)} = (7,0004; -2,99997; -7,00001; -10)^T$

3 итерация:

$x_1^{(3)} = \frac{1}{67} \cdot (-4 \cdot (-2,99997) - 5 \cdot 10 + 507) \approx 6,999998$

$x_2^{(3)} = \frac{1}{115} \cdot (-9 \cdot (-7,00001) - 6 \cdot 10 - 348) \approx -2,999999$

$x_3^{(3)} = \frac{1}{86} \cdot (10 \cdot (-3) - \cdot 10 - 562) \approx -7$

$x_4^{(3)} = \frac{1}{66} \cdot (2 \cdot 7 - 8 \cdot 3 - 3 \cdot 7 - 629) \approx -10$

$x^{(3)} = (6,999998; -2,999999; -7; -10)^T$

\begin{equation*}
    r^{(0)} = Ax^{(0)} - b =
    \begin{pmatrix}
        544,790 \\
        -350,087 \\
        -552,041 \\
        -629,011 \\
    \end{pmatrix}
    -
    \begin{pmatrix}
        507 \\
        -348 \\
        -562 \\
        -629 \\
    \end{pmatrix}
    =
    \begin{pmatrix}
        37,790 \\
        -2,087 \\
        9,959 \\
        -0,011 \\
    \end{pmatrix}
    , \|r^{(0)}\| = 37,790
\end{equation*}

\begin{equation*}
    r^{(1)} = Ax^{(1)} - b =
    \begin{pmatrix}
        507,174 \\
        -348,923 \\
        -562,005 \\
        -629,005 \\
    \end{pmatrix}
    -
    \begin{pmatrix}
        507 \\
        -348 \\
        -562 \\
        -629 \\
    \end{pmatrix}
    =
    \begin{pmatrix}
        0,174 \\
        -0,923 \\
        -0,005 \\
        -0,005 \\
    \end{pmatrix}
    , \|r^{(1)}\| = 0,923
\end{equation*}

\begin{equation*}
    r^{(2)} = Ax^{(2)} - b =
    \begin{pmatrix}
        507,02692 \\
        -347,99664 \\
        -562,00116 \\
        -629,00101 \\
    \end{pmatrix}
    -
    \begin{pmatrix}
        507 \\
        -348 \\
        -562 \\
        -629 \\
    \end{pmatrix}
    =
    \begin{pmatrix}
        0,02692 \\
        -0,00346 \\
        -0,00116 \\
        -0,00101 \\
    \end{pmatrix}
    , \|r^{(2)}\| = 0,02692
\end{equation*}

\begin{equation*}
    r^{(3)} = Ax^{(3)} - b =
    \begin{pmatrix}
        506,99987 \\
        -347,99989 \\
        -562,00001 \\
        -629,000001 \\
    \end{pmatrix}
    -
    \begin{pmatrix}
        507 \\
        -348 \\
        -562 \\
        -629 \\
    \end{pmatrix}
    =
    \begin{pmatrix}
        -0,00013 \\
        -0,00011 \\
        -0,00001 \\
        -0,000001 \\
    \end{pmatrix}
    , \|r^{(3)}\| = 0,00013
\end{equation*}

$$\frac{\|r^{(0)}\|}{\|r^{(3)}\|} = \frac{37,790}{0,00013} \approx 290692$$

\newpage

\item Используя апостериорную оценку, вычислим погрешность приближенного решения, полученного на третьей итерации каждого метода.

Метод Якоби:

\begin{equation*}
    \|x^{(2)} - x^*\| \leq \frac{\|B\|_\infty}{1 - \|B\|_\infty} \cdot \|x^{(2)} - x^{(1)}\|_\infty = \frac{0,1969}{1 - 0,1969} \cdot 
    \begin{Vmatrix}
        -0,034 \\
        -0,014 \\
        -0,003 \\
        -0,035 \\
    \end{Vmatrix}_\infty
    = 0,0086
\end{equation*}

Метод Зейделя:

\begin{equation*}
    \|x^{(2)} - x^*\| \leq \frac{\|B_2\|_\infty}{1 - \|B_1\|_\infty} \cdot \|x^{(2)} - x^{(1)}\|_\infty = \frac{0,1343}{1 - 0,1969} \cdot 
    \begin{Vmatrix}
        -0,003 \\
        0,008 \\
        -0,001 \\
        -0,001 \\
    \end{Vmatrix}_\infty
    = 0,0013
\end{equation*}

\end{enumerate}

\end{document}